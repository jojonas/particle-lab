% !TeX encoding = UTF-8
% !TeX spellcheck = en_US

\documentclass[
	paper=A4,
	parskip=full,
	chapterprefix=true,
	11pt,
	headings=normal,
	bibliography=totoc,
	listof=totoc,
	titlepage=on,
]{scrreprt}

\usepackage{../../lieb}
\usepackage{rotating}

\usepackage{feynmp}
\DeclareGraphicsRule{.1}{mps}{*}{}
\graphicspath {{../images/}}

\usepackage{isotope}

\heads{RWTH Aachen \\ Particle Physics Lab}{T8 \\ Positron Lifetime in Matter}{Lieb | Stettner \\ \today} 
\date{\today}

\newcommand{\thirdwidth}{0.32\textwidth}
\newcommand{\halfwidth}{0.48\textwidth}
\newcommand{\fullwidth}{1.0\textwidth}

\setlength\parindent{0pt}
\setlength{\parskip}\medskipamount

\title{Particle Physics Laboratory Class \\ \quad \\ Experiment T8 | Positron Lifetime in Matter}
\author{Jonas Lieb (312136) \\ Jöran Stettner (312169) \\ \\  RWTH Aachen}



\begin{document}

\maketitle

\cleardoublepage

\setcounter{tocdepth}{2}
\tableofcontents

\cleardoublepage

\chapter{Introduction and Theory}

In this laboratory report, a measurement of the positron lifetime in different materials is presented. The experiment uses the special properties of the positron source: Subsequent to the $\beta^+$-decay of $\isotope[22]{Na}$, the decay product $\isotope[22]{Ne}$ emits a high energy photon which can be used as start signal of the time-measurement. 

\section{Positrons in Matter}
The positron as elementary particle is stable in vacuum but behaves different in matter. It can interact electromagnetically and finally annihilate with an electron to photons. \\
At first, the incoming positrons interact with the electrons of the material and are slowed down. The stopping process can be described by the Bethe-Bloch equation and yields in total to the following time values:

\begin{table}[htbp]
	\centering
	\begin{tabular}{ 
			l
			l
			}
		\toprule
		{Material} & {Stopping Time [$\si{\pico\second}$]} \\ 
		\midrule
		Polyethylen & $\approx 300.22 $ \\
		Aluminum &  $\approx 3.11 $ \\
		\bottomrule
	\end{tabular}
	\caption{Time-values for the stopping of a $\SI{180}{\kilo\electronvolt}$ positron, values from \cite{Lab_manual_T8}. Dominating is the time to slow down to thermal energies $E\approx \SI{0.025}{\electronvolt}$.}
	\label{tbl:stop_times}
\end{table}

\subsection{Positrons in Metals}
Depending on the properties of the surrounding material, it is also possible for a positron to form a bound state with an electron, the so called positronium. The conditions can be understood in the Oere-gap model and show that positrons in metal do not form bound states (too large electron-density)\cite{Lab_manual_T8}. In metals, the lifetime of an incoming positron is therefore expected in the $\si{\pico\second}$ regime.

\subsection{Positrons in Plastic and the Decay of Positronium}

In polyethylen on the other hand, the formation of positronium is possible. This bound state exists in two spin configurations: A singlet state (electron and positron spin add up to zero) and a triplet state (spin-sum equals 1). The decay of the two positronium states differ, see figure \ref{fig:feynman_positronium}. Since these electromagnetic processes conserve Parity, the singlet state has to decay to an even number of photons and the triplet state to an odd number\cite{Lab_manual_T8}. \\
Furthermore, the conversion between the two positronium states is possible in matter (e.g. exchange of the electron). This process is more likely and therefore faster than the decay of the triplet state. Concluding, in plastic the formation of positronium is expected which subsequently decays to two photons either directly (singlet state, short lifetime)  or after a conversion (triplet state, longer lifetime/conversion time).

\begin{figure}
	\centering
	\includegraphics{feynman_positroniumdecay} \\
	\caption{Feynman Diagrams of the Annihilation processes of a positron with an electron (from the surrounding material of from the bound state in positronium),taken from \cite{Lab_manual_T8}. In presence of a nucleus which absorbs an outgoing photon, the annihilation to one photon is possible as well (suppressed).}
	\label{fig:feynman_positronium}
\end{figure}


\chapter{Experimental Setup}

The measurement of the lifetime is based on the property of the positron source $\isotope[22]{Na}$ which emits nearly instantaneously a photon in the subsequent decay ($\isotope[22]{Ne}^*\rightarrow \isotope[22]{Ne} $). The idea is to use this gamma ray with an energy of $E = \SI{1.2}{\mega\electronvolt}$ as start signal for the time measurement and one of the photons from the positron annihilation as stop signal ($E=\SI{511}{\kilo\electronvolt}$). The high energy photons are detected with plastic scintillators where they mostly create a high energy compton electron which subsequently produces scintillation light. Next to the scintillators on each side of the sample, two photomultiplier tubes (PMTs) are installed where the scintillation photons are converted to an electrical signal. The PMTs provide two signals: One at the last dynode and finally an integrated signal from the anode. The detector setup is schematically shown in figure \ref{fig:positron_setup}. Additional to one \isotope[22]{Na} source in Aluminum and one in Polyethylen, a \isotope[60]{Co} source (two coincident $\gamma$) is used to measure the resolution of the experimental setup. Furthermore, the two detectors can be replaced by a pulse generator for time calibration.

\begin{figure}
	\centering
	\includegraphics{positron_setup}
	\caption{Scheme of the experimental setup, taken from \cite{Lab_manual_T8}.}
	\label{fig:positron_setup}
\end{figure}

The two signals from each PMT are processed in a coincidence circuit which consists of different modules to reject background. The full electrical circuit is shown in figure \ref{fig:full_circuit}. The processing of the anode signal consists of the following steps (outer lines): At first, the signals are amplified. To reject background, the signal height is afterwards compared to a threshold in a Constant Fraction Discriminator (CFD) which provides a NIM-pulse and prevents 'time walk'\cite{Lab_manual_T8}. The actual measurement of the time difference between the start and the stop photons takes place in the Time Amplitude Converter (TAC) which charges a capacitor for the time between the two signals and delivers a pulse after discharging over a resistor. Finally, the pulse height is analyzed in a Multi Channel Analyzer module (MCA) and filled in a histogram. Parallel to this, the signals from the last dynode are processed and used to tune the setup for the photons of interest: The dynode signal is proportional to the energy of the initial photon and can thus be used to reject events where the energy does not match the expectation. The signal processing after the two PMTs is therefore specialized: One circuit accepts signals as expected from the first initial photon ($\gamma$ from \isotope[22]{Ne}) and the other from the annihilation photons ($e^+e^- \rightarrow 2\gamma$). The adjustment of these energy windows which are accepted is described in section \ref{sec:PrepandCal}. If both dynode signals pass the Window Discrimators (WD), they are used as coincidence signal and trigger the gate of the MCA. Only in this case, the pulse from the TAC is filled in the histogram. 

\begin{figure}
	\centering
	\includegraphics{aufbau}
	\caption{Full electrical circuit showing the processing of the two anode and dynode signals. The outer lines describe the steps of the anode signals which result in the time-measurement with the TAC and MCA. The inner lines show the processing of the dynode signals which are used to reject events which do not match the energy expectation of the start or stop photons.}
	\label{fig:full_circuit}
\end{figure}


\chapter{Preparation and Calibration of the Circuit}
The setup as presented in the last chapter has to be adjusted to optimize the background rejection and to synchronize the delay of different parts of the full electrical circuit. These adjustment steps and the final time calibration with a pulse generator are presented in this chapter.
\section{Adjustment of the Window Discriminators}

To choose the window of accepted energies (dynode signal height), the circuit of the dynode is altered such that the MCA measures the energy spectrum of the detected photons, see figure \ref{fig:WD_circuit}. If the amplitude of the amplified dynode signal passes the window discrimator, the value is analyzed and added to the energy spectrum. 
 
\begin{figure}
 	\centering
 	\includegraphics{aufbau_wd}
 	\caption{Altered circuit of one dynode signal to adjust the window discriminator.}
	\label{fig:WD_circuit}
\end{figure}
 
If the discrimination is disabled, the energy spectrum shows the typical form of a spectrum from a scintillation detector with the characteristic Compton edge. The WD of the start circuit has to be tuned to the high energy photon and the WD of the stop circuit to the annihilation photons. The windows are therefore chosen around the corresponding Compton edges. The choice of the parameters of the WDs are presented in figures \ref{fig:WD_start} and \ref{fig:WD_stop} and listed in table \ref{tbl:WD_values}. The values are not adjusted very tight because following measurements with a Cobalt source have to be conducted with the same setup.

\begin{table}[htbp]
	\centering
	\begin{tabular}{ 
			l
			l
			l
			l
		}
		\toprule
		{Circuit} & {Threshold} & {Window} & {Time Constant} \\ 
		\midrule
		Start & $185$ & $450$ & $\SI{1}{\micro\second}$ \\
		Stop & $0$ & $280$ & $\SI{1}{\micro\second}$ \\
		\bottomrule
	\end{tabular}
	\caption{Adjusted parameters of the two Window Discriminators, one tuned for the start photon and the other for the annihilation photons.}
	\label{tbl:WD_values}
\end{table}

\section{Adjustment of the Constant Fraction Discriminators}
\section{Synchronization of the Gate and the TAC Signal}

\section{Time Calibration and Determination of the Electronic Resolution}
The goal of this calibration experiment is to map the MCA channels to a physical delay between the PMT signals. 

To simulate realistic conditions, the experimental setup is very similar to the one depicted in figure \ref{fig:full_circuit}. The only  exception is that the detectors as well as the radioactive source are replaced by a digital pulse generator. The pulse generator is configured such that the pulse heights and lengths resemble the signals from the PMTs.  
The delay is manually entered into the pulse generator, delaying the stop circuit. A delay range of \SIrange{0}{20}{\nano\second} is sampled in \SI{1}{\nano\second} steps, additional points at \SI{25}{\nano\second} and \SI{30}{\nano\second} are recorded.

\begin{sidewaysfigure}
	\centering
	\includegraphics{calibration_peaks}
	\caption{Peaks of the time calibration. Each peak has been normalized to an area of \num{1}.}
	\label{fig:calibration_raw}
\end{sidewaysfigure}

The measured data is shown in figure \ref{fig:calibration_raw}. As expected, the measurements show up as narrow equidistant peaks. Each peak has been normalized to contain \num{1} event in total. This way the peak shapes can be compared. As one can easily see (especially around \SIrange{4}{13}{\nano\second}), the peaks are actually made up from two separate Gaussian curves. By exchanging modules from the experimental setup this behavior can be traced back to the \SI{16}{\nano\second} delay between the stop-circuit-CFD and the TAC. Replacing the module by a different delay does not mitigate the issue.

Since neither of the bumps is physically motivated, both will be considered in the calibration and the uncertainty adjusted accordingly.
First the extend of a peak is calculated. Around a manually entered seed, the maximal bin value is calculated. All adjacent bins with more than \SI{2}{\percent} of the maximum value are assumed to belong to the peak. From the entire range, the centroid is calculated. This is the center value $\bar{x}$ of the curve. Symmetrically around the center value $\bar{x}$, a region from $\bar{x}-\sigma_x$ to $\bar{x}+\sigma_x$ is defined. $\sigma_x$ is initially set to \num{0}, but then increased until the integral of the region contains \SI{68}{\percent} of the total peak value ($ = \num{1}$). The value $\sigma_x$ can then be interpreted as uncertainty, analogous to the standard deviation of a Gaussian distribution.

Each peak is now characterized by a delay time $\Delta t$, a center channel value $\bar{x}$ and a peak uncertainty $\sigma_x$.
A linear regression is performed on this data. Using $\chi^2$ minimization, the parameters of $f(x) = a x + b$ are optimized. 
The uncertainty of the parameters is determined by simulating "pseudo-experiments": \num{1000} times, all data points are chosen from a normal distribution around their measured value with their uncertainty. The fit is performed for each pseudo-experiment, mean and standard deviation of the fit result yield the final value and uncertainty on the fit parameters.

\begin{figure}
	\centering
	\includegraphics{calibration}
	\caption{Linear regression result.}
	\label{fig:calibration_linreg}
\end{figure}

The regression result with the residuum is depicted in figure \ref{fig:calibration_linreg}. The residuum plot on the bottom contains the channel uncertainty projected onto the fitted function. Despite a small visible systematic deviation in the residuum plot, $\chi^2/\mathrm{ndof} = 1.74$ indicates that the fitted model describes the data. This explicitly justifies the choice of the delay-value in the stop-circuit since TAC and MCA seem to operate in a linear region.

The final calibration function is
\begin{equation}
	\Delta t = \left[\left(\num{0.0023292 \pm 0.0000034}\right) \cdot \mathrm{channel} + \left(\num{-5.163 \pm 0.025}\right)\right] \si{\nano\second}
\end{equation}

This result can be combined with the mean and sample standard deviation of the obtained peak widths to calculate the pure electronic resolution:
\begin{equation}
	\sigma_{t,\mathrm{elec}} = \SI{64.4 \pm 15.5}{\pico\second}
\end{equation}

\chapter{Conduction and Analysis}

\chapter{Systematic Uncertainties}

\chapter{Results and Conclusion}


\cleardoublepage

\bibliographystyle{utphys}
\bibliography{T8_bib}{}

\end{document}


