% !TeX encoding = UTF-8
% !TeX spellcheck = en_US

\documentclass[
	paper=A4,
	parskip=full,
	chapterprefix=true,
	11pt,
	headings=normal,
	bibliography=totoc,
	listof=totoc,
	titlepage=on,
]{scrreprt}

\usepackage{../../lieb}

\usepackage{feynmp}
\DeclareGraphicsRule{.1}{mps}{*}{}

\graphicspath {{out/}{bilder/}{data/}}
\heads{RWTH Aachen \\ Particlephyics Lab}{T17 \\ W-Boson}{Lieb | Stettner \\ \today} 
\date{\today}

\newcommand{\MET}{\ensuremath{{\slashed{E}_\mathrm{T}}}\xspace}
\newcommand{\ELET}{\ensuremath{{E_\mathrm{T}^\mathrm{el}}}\xspace}

\newcommand{\thirdwidth}{0.32\textwidth}
\newcommand{\halfwidth}{0.48\textwidth}
\newcommand{\fullwidth}{1.0\textwidth}

\setlength\parindent{0pt}
\setlength{\parskip}\medskipamount

\title{Particle Physics Laboratory Class \\ \quad \\ Experiment T17 | W-Boson }
\author{Jonas Lieb, (312136) \\ Jöran Stettner (312169) \\ \\  RWTH Aachen}

\newcommand{\dnull}{D$\slashed{\mathrm{0}}$\xspace}

\begin{document}

\maketitle

\cleardoublepage

\setcounter{tocdepth}{2}
\tableofcontents

\cleardoublepage

\chapter{Introduction}

This analysis deals with \PW-Boson physics conducted at the \dnull experiment at the Tevatron accelerator (FERMILAB, Chicago). The data has been taken in proton-antiproton collisions at a center-of-mass energy of $\sqrt{s} = \SI{1960}{\giga\electronvolt}$.

\section{Units}
In this report, a natural unit system of particle physics is used: first, the speed of light and Planck's constant are fixed:
\begin{equation}
c \defeq \num{1},\quad \hbar \defeq \num{1}
\end{equation}
From this convention, many quantities arise in units of energy. Additionally, the gigaelectronvolt (\si{\giga\electronvolt}) is chosen as basic energy unit, in order to deal with numbers of magnitude $\order{1}$.

\subsection{\PW-Boson Production and Decay}
The process of interest is the \PW-Boson production from the valence quarks of the protons and antiprotons, and the decay into an electron and an electron neutrino.
\begin{equation}
	\Pproton\APproton \rightarrow \PWminus \rightarrow \Pelectron \APnue
\end{equation}
Equivalently, there exists a oppositely charged process:
\begin{equation}
\Pproton\APproton \rightarrow \PWplus \rightarrow \Ppositron \Pnue
\end{equation}
With the assumption that the \PWplus and \PWminus have similar properties (except the electric charge), one does not expect any difference between the processes. Because of that, the analysis will not differentiate between them and treat positrons as electrons in the final state.

Both processes combined have an expected cross section\cite{HBK+2013Experiment} of 
\begin{equation}
	\sigma = \SI{2.58 +- 0.09}{\nano\barn}
\end{equation}
and should show a resonance at the predicted \PW-Boson mass\cite{Oo2014Review} of 
\begin{equation}
	m_{\PW} = \SI{80.385 +- 0.015}{\giga\electronvolt}
\end{equation}

\section{The \dnull Detector}
The measurement took place in 2004 until 2006 at the \dnull experiment at FERMILAB. The \dnull detector is a cylindrical particle detector located around the Tevatron beam pipe. It consist of a silicon tracker in the center region, enclosed by an Argon electromagnetic calorimeter.

Inside the detector, a Cartesian coordinate system and a cylindrical coordinate system are used. The origin of both systems lies in the interaction point, with the z-axis pointing along the beam pipe. The angle $\phi$ is measured in a plane perpendicular to the z-axis. 

Because the colliding protons are composite particles, the longitudinal momentum of the interacting quarks is unknown. The transversal components, however, are zero in the initial state. From momentum conservation it follows that the sum of all transversal momenta in the final state is also zero. 
To represent this conservation in the transversal plane, energies and momenta are projected onto it. This introduces the transverse electron energy \ELET.
Another important definition is the missing transverse energy \MET. 


\chapter{Monte-Carlo Samples}

\cleardoublepage

\bibliographystyle{utphys}
\bibliography{T17_bib}{}

\end{document}
